\documentclass[a4paper,12pt]{article}
% Только для XeLaTeX/LuaLaTeX, для pdfLaTeX — см. выше
\usepackage{polyglossia}
\setdefaultlanguage{russian}
\usepackage{geometry}
\usepackage{enumitem}
\usepackage[normalem]{ulem}
\usepackage{amsmath}
\usepackage[table]{xcolor}
\usepackage{tikz}
\usepackage{titlesec}
\usepackage{setspace}
\usepackage{fontspec}
\setmainfont{Times New Roman}

\geometry{top=1.5cm, bottom=1.5cm, left=1.5cm, right=1.5cm}

\titlespacing*{\subsubsection}{1.5cm}{*}{*}
\titlespacing*{\subsection}{0.5cm}{*}{*}

\titleformat{\subsection}
  {\normalfont\normalsize\color{black}}
  {\thesubsection}{0.5em}{}

\titleformat{\subsubsection}
  {\normalfont\normalsize\color{black}}
  {\thesubsubsection}{0.5em}{}

\definecolor{amethyst}{rgb}{0.6, 0.4, 0.8}

\begin{document}
\begin{center}
    \Large{\textbf{Договор №{{number_dogovor}} }} \\
    \large{\textbf{возмездного оказания репетиторских услуг}}
\end{center}
Заключен \hfill {{data_dogovor}} г.\\[1cm]
\noindent
Гражданин Российской Федерации, Федосцев Степан Александрович, зарегистрированный в качестве самозанятого лица и являющийся налогоплательщиком с 03.09.2023 (ИНН: 661218987720), далее именуемый <<Преподаватель>>, с одной стороны,
и \underline{ {{customer_full_name}} } (документ, удостоверяющий личность: \underline{паспорт}), \textbf{}далее именуем(ый/ая) <<Заказчик>>, с другой стороны, далее совместно именуемые <<Стороны>>, заключили настояйщий договор, далее именуемый <<Договор>>, о нижеследующем.

\Large{\section{Предмет договора}}
\subsection{Преподаватель оказывает репетиторские услуги лично, своими силами, без привлечения третьих лиц, по обучению ученика, которым является \underline{ {{student_name}} }, далее именуем(ый/ая) <<Ученик>>.}
\subsection{Обучение Ученика проводится в форме индивидуальных занятий, далее именуемых <<Занятия>>. Занятия будут проводиться в онлайн-формате через согласованную сторонами платформу.}
\subsection{Услуги оказываются в форме индивидуальных уроков продолжительностью 90 минут в виде онлайн-созвона Преподавателя с Учеником.}
\subsection{Преподаватель обязуется предоставить Ученику услуги по математике для подготовки к Государственной итоговой аттестации, а Заказчик — принять и оплатить их, согласно условиям, определяемым Договором..}
\subsection{Преподаватель гарантирует максимальное качество предоставляемых образовательных услуг в пределах своей компетенции.}
\subsection{Преподаватель не гарантирует получения Учеником какого-либо конкретного результата и не несёт ответственности в случае недостижения какого-либо конкретного результата.}
\subsection{Выдача Сертификата или иного документа установленного образца по окончании Занятий не предусмотрена.}
\newpage
\Large{\section{Общие условия Занятий}}
\subsection{Цель Занятий: \uline{ {{target}} }}
\subsection{Занятия проходят онлайн в формате созвона с применением следующих онлайн-инструментов: Google Meet, Miro и других аналогичных.}
\subsection{Период занятий - с ${{data_dogovor}}$ по $01.06.2026$.}
\subsection{Длительность одного Занятия составляет полтора астрономических часа (90 минут).}
\subsection{О днях и времени Занятия Преподаватель и Ученик договариваются через менеджер Telegram.}
\subsection{Расписание Занятий для Ученика согласовывается с Преподавателем и Учеником в устной/письменной форме и может быть скорректировано в ходе последующих Занятий.}
\subsection{Ученик действует от имени Заказчика при согласовании с Преподавателем расписания, длительности и содержания последующих Занятий.}
\Large\section{Стоимость и оплата Занятий}
\subsection{Заказчик вносит 100\% (полную) предоплату за месяц Занятий, предстоящих Ученику по расписанию.}
\subsection{Стоимость одного Занятия длительностью $90$ минут равна $1500$ (одна тысяча пятьсот) рублей.}

\subsection{Оплата производится Заказчиком (Может быть произведена Учеником при согласии Заказчика) за безналичный расчет по реквизитам, указанным ниже в Договоре в пункте 4.}
\subsection{Факт очередной предоплаты является подтверждением того, что на текущий момент соответствующие репетиторские услуги были оказаны Преподавателем в полном объеме и в надлежащем качестве, а Заказчик и Ученик не имеют претензий к Преподавателю}
\Large\section{Реквизиты для оплаты}
\subsection{Данные для оплаты:}
\normalsize
\begin{itemize}
    \item Оплата производится через:\hfill Систему Быстрых Платежей (СПБ)
    \item Получатель: \hfill Федосцев Степан Александрович
    \item Номер получателя: \hfill +7 (902) 271-81-09
    \item Банк получателя: \hfill АО <<ТБанк>>
\end{itemize}
\subsection{Оплату необходимо произвести в течение $24$ часов с момента согласования расписания на месяц. В случае неоплаты в указанный срок проведение Занятий приостанавливается до момента поступления оплаты.}
\subsection{После получения оплаты Преподаватель обязуется выдать Заказчику чек в электронном виде через приложение <<Мой налог>> в течение одного рабочего дня и направить его Заказчику.}
\newpage
\Large\section{Пропуск Занятия, переносы  и опоздания}
\subsection{Стороны согласны, что залогом успеха в учёбе является регулярность Занятий, а также сознают, что отмены Занятий создают организационные проблемы, поэтому Стороны обязуются свести к минимуму отмены Занятий, а также заблаговременно предупреждать об отменах.}
\subsection{Об отмене Занятия Заказчик и Преподаватель уведомляют друг друга не позднее 20:00 дня, предшествующего дню проведения Занятия.}

\subsection{Если Ученик пропустил Занятие, он обязан незамедлительно проинформировать об этом Преподавателя, предоставив объяснение и/или подтверждающие документы.}

\subsection{Признание причины уважительной и предоставление права на перенос Занятия осуществляется по усмотрению Преподавателя. В случае, если пропуск не признан уважительным, стоимость Занятия возврату не подлежит, и перенос не осуществляется.}

\subsection{В случае опоздания Ученика, время занятия не продлевается.}

\subsection{При отмене или пропуске Занятия по вине Преподавателя (включая болезнь, форс-мажор и другие обстоятельства), Преподаватель обязан согласовать с Заказчиком новую дату и время проведения соответствующего Занятия.}


\subsection{Факт отмены/переноса Занятия не оказывает влияния на срок очередной предоплаты в соответствии с расписанием Занятий.}
\Large\section{Условия расторжения Договора}
\subsection{По истечении срока действия, установленного в настоящем Договоре, Договор считается прекращённым автоматически, без необходимости оформления дополнительного соглашения о его расторжении и/или уведомления об этом другой Стороны.}

\subsection{Заказчик вправе в одностороннем порядке отказаться от исполнения Договора, уведомив Преподавателя не менее чем за \underline{14} календарных дней до предполагаемой даты расторжения. При этом Стороны обязаны исполнить обязательства, возникшие к моменту расторжения, а Преподаватель, в случае получения предоплаты, обязан возвратить Заказчику сумму, пропорциональную стоимости неоказанных услуг, в течение \underline{14} банковских дней с момента прекращения действия Договора.}

\subsection{Преподаватель вправе досрочно расторгнуть Договор в случае, если Ученик систематически нарушает дисциплину либо своим поведением препятствует нормальному ходу Занятий.}

\subsection{Любая из Сторон может потребовать расторжения Договора в случае, если становится очевидной невозможность достижения Учеником целей, указанных в разделе 2.1 Договора.}


\Large\section{Прочие условия}
\subsection{Договор действует с момента подписания до \underline{1 июня 2026 года}.}
\subsection{Договор составлен и подписан в электронном виде, его экземпляры хранятся у каждой из Сторон.}
\subsection{По всем вопросам, не урегулированным в Договоре, Стороны руководствуются действующим законодательством Российской Федерации.}
\newpage
\Large\section{Подписи сторон}
\normalsize
    \subsection{Стороны подтверждают, что условия настоящего Договора поняты и приняты в полном объеме.}

    \subsection{Стороны согласны, что Договор может быть подписан в следующем порядке:}
        \subsubsection{Преподаватель направляет Заказчику проект Договора с заполненными своими реквизитами.}
        \subsubsection{Заказчик вносит свои персональные данные в соответствующие поля, подписывает документ собственноручно либо с помощью электронного средства подписи.}
        \subsubsection{Подписанный Заказчиком экземпляр направляется Преподавателю в виде отсканированной копии или в PDF-формате через электронную почту либо Telegram.}
        \subsubsection{После получения Преподаватель также проставляет свою подпись и направляет Заказчику подписанный экземпляр.}

    \subsection{Допускаются следующие способы подписания:}
        \subsubsection{Преподавателем — с использованием усиленной квалифицированной электронной подписи (в том числе через сервисы, такие как Контур.Крипто) либо собственноручной подписи с отправкой скан-копии.}
        \subsubsection{Заказчиком — с использованием собственноручной подписи и последующей передачей отсканированной копии или цифрового варианта, либо путём использования приложений для подписания (например, Adobe Sign, Fill \& Sign).}
    \subsection{Документы, подписанные указанными способами и переданные по электронной почте либо в мессенджерах, признаются равнозначными оригиналам и имеют юридическую силу.}
    \vspace{1cm}
\begin{tabular}{p{0.45\textwidth} p{0.45\textwidth}}
    \Large\textbf{Преподаватель:} & \Large\textbf{Заказчик:} \\[0.6cm]
    \textbf{ИНН:} 661218987720 & \textbf{Паспорт:} \\[0.2cm]
    & \hspace{0.7cm} серия: \uline{ {{passport_series}} } \qquad номер: \uline{ {{passport_number}} } \\[0.3cm]
    & \hspace{0.7cm} выдан: \uline{ {{passport_issue}} } \\[0.3cm]
    & \hspace{0.7cm} дата выдачи: \uline{ {{passport_issue_date}} } \\[0.3cm]
    & \hspace{0.7cm} код подразделения: \uline{ {{passport_dept_code}} } \\[0.3cm]
    & \hspace{0.7cm} адрес регистрации: \uline{ {{registration_address}} } \\[0.2cm]
    \textbf{Telegram:} \underline{\text{@st\_fedostsev}} & \textbf{Telegram:} \uline{ {{telegram}} } \\[0.3cm]
    \textbf{Email:} \underline{\text{st.fedostsev@gmail.com}} & \textbf{Email:} \uline{ {{email}} } \\[1.0cm]
    \rule{6.4cm}{0.4pt} & \rule{7.8cm}{0.4pt} \\[0.5cm]
    / \underline{Федосцев Степан Александрович} / & / \uline{ {{customer_full_name}} } /
\end{tabular}

\end{document}
